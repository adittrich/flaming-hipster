\chapter{Gossiping in P2P Networks} 
\label{chap:gossiping}


\section{Short Introduction to Peer-to-Peer Networks}
\subsection{What is Peer-to-Peer?}
% What is it?
% What is the basic idea?
Traditional services in the Internet follow the client-server model where a
server entity provides a service or resource which then is used by
possibly multiple client entities. Peer-to-Peer networks use a completely
different approach. Ideally every participating entity acts as a server and a
client simultaneously. Every participating entity in such a network is treated
equally and is sometimes also called Servent, a word created by combining the
words server and client. A Servent does not only use the service or resource
the network provides but is also participating in providing it to the other
Servents. The main prerequisite for such an architecture is that the service or
resource provided by the network is distributable. Other requirements are
powerful servents interconnected with a suitable upload and download bandwidth.
\\% What are typical applications?
Applications of Peer-to-Peer include very popular ones like file
sharing, instant messaging or video calls over the Internet. Other applications
are shared computing, shared workspace environments or information
dissemination, e.g. software updates.
\\% What are possible Applications?
% What are dis-/advantages over the server-client model?
The Peer-to-Peer design gives several advantages over the client-server model.
In a pure Peer-to-Peer designed network there is no need for a central entity
which makes the network independent of a single point of failure which can also
act as a bottleneck. While in a client-server architecture a growing number of
participants will exceed the resources provided by a central server, in a
Peer-to-Peer network the resources grow with the number of participants. The
lack of a central entity also means that the service can be provided ad-hoc with
the combined resources that each participant contributes to the network without
the need of a dedicated server. Another advantage can be the better
usage of bandwidth. If an information has a single point of entry,
e.g. a video live stream or a software update and a client-server model is used
the information has to be sent multiple times over the same links clogging the
bandwidth with repetitive data. A Peer-to-Peer network could organise this
multicast communication in a way that avoids unnecessary data transmissions. As
mentioned before the Peer-to-Peer approach eliminates the single point of
failure a central server is which results in a more reliable service. More
specifically an attack or a failure of a central server disables the service
entirely while a failing servent has minor effects on the whole service.
\\
The next subsection will discuss and classify different approaches in Peer-to-Peer
structures.
\subsection{Types of Peer-to-Peer Architectures}
Several applications using the Peer-to-Peer paradigm have been proposed and used
within the last three decades. While offering a solution for different services
or resources they also can be classified by the organization of the network
itself. The main difference in architecture being how they organize the
resources or services and handle requests and responses. This subsection will
shortly introduce three main classes of Peer-to-Peer architectures.
\subsubsection{Unstructured Peer-to-Peer}
\paragraph{Centralized Peer-to-Peer}
The first popular Peer-to-Peer based application was Napster. It used a
central Server for storing the information where to find which file in the
network. After getting an answer from the central server a servent could then
directly download the desired file from the origin. This approach can be
generalized as follows. A central server is used for organisation
of the service or resource provided by the network. The requestet service or
resource is then provided directly. This architecture is called a
Centralized Peer-to-Peer network.
\\
The main disadvantage is the central server since it contradicts most of the
 advantages of the Peer-to-Peer architecture, as it acts as a single point of 
failure and is the bottleneck of such a network.
\paragraph{Pure Peer-to-Peer}
In a pure Peer-to-Peer approach instead of managing the information about the
networks resources or services centrally, every servent manages its resources
locally. Request are send over the network using flooding. A Servent sends a
time limited request to all its known neighbors which repeat this action until
the servent able to answer the request is found or the request expires. This
method clearly produces a lot of redundant network traffic which highly depends
on the topology of the network. A Peer-to-Peer network constructed in such a way
eliminates the need of a central entity is locally organized but has its
bottleneck in the bandwidth which would have to grow exponentially with the
number of participants and requests. The way servents decide which other
participants to select as neighbors is most crucial to the performance of this
type of Peer-to-Peer networks. The resulting topology's properties have to
match the purpose of the network. The main challenge in constructing such a
network is finding a topology which keeps network traffic at a minimum while not
loosing the decentralized way of network creation.
\paragraph{Hybrid Peer-to-Peer}
There are also applications using a mix of the previously introduced
Peer-to-Peer architectures. The main reason for this approach is to reduce
the number of messages while still using an unstructured approach. This is
achieved through introducing hierarchies into the network. Superpeers or
ultrapeers are selected dynamically. They are responsible for a number of simple
servants and connected to other superpeers in an unstructured manner. The complexity of such a network
is reduced depending on the number of simple servents the superpeers are
responsible for. There still are no single points of failure although a failure
of a superpeer has a bigger impact on the network than of a servant in a pure
Peer-to-Peer network. The superpeers can be selected depending on availability,
stability, bandwidth or any other aspect serving the purpose of the network.
\subsubsection{Structured Peer-to-Peer}
While Cyclon is meant for use in unstructured Peer-to-Peer networks this chapter
would be incomplete without mentioning another approach to construct such
networks. Structured Peer-to-Peer networks try to eliminate the cost of
flooding requests while still organising the service decentrally. While in
centralized Peer-to-Peer applications the information about the service is
stored centrally this information is distributed among the servents in the whole
network. This is achieved by distributed hash tables (DHTs). First for every
piece of information a hash value is calculated which determines which of the
servents will be responsible for this information. The servents form a ring
topology. Every servent has one neighbor that is responsible for all
information with smaller hash values and the other for all the information with greater hash
values than hash values of the information the servent is responsible for
itself. The resulting chain is made to a ring by connecting the servents
responsible for the maximum and minimum hash values. A request is send to an
arbitrary servent on the ring and then forwarded to the servent believed to be
closest to the hash value of the information until the servent with the
information si found. This approach clearly greatly depends on the function
calculating the hash values, so that the information is equally distributed
among the servents. In a network with a high churn rate this approach possibly
needs to adjust its hash function to eliminate unequal distribution. This may
produce a lot of overhead compared to unstructured Peer-to-Peer networks.


\section{Overlay Topologies}
%What is this? 
As discussed above the main challenge in pure Peer-to-Peer networks is finding a
suitable topology for the purpose of a network. Some of the desired 
properties are shown in Chapter \ref{chap:properties} 

\section{Gossiping in P2P}
%What is gossiping in general?  

\section{Cyclon} 
%How does Cyclon work? Pseudocode of shuffling algorithms