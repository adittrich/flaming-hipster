\chapter{Gossiping in P2P Networks} 
\label{chap:gossiping}


\section{Short Introduction to Peer-to-Peer Networks}
\subsection{What is Peer-to-Peer?}
% What is it?
% What is the basic idea?
Traditional services in the Internet follow the client-server model where a
server entity provides a service or resource which then is used by
possibly multiple client entities. Peer-to-Peer networks use a completely
different approach \cite{Wehrle2005}. Ideally every participating entity acts as
a server and a client simultaneously. Every participating entity in such a network is treated
equally and is sometimes also called servent, a word created by combining the
words server and client. A servent does not only use the service or resource
the network provides but is also participating in providing it to the other
servents. The main prerequisite for such an architecture is that the service or
resource provided by the network is distributable. Other requirements are
powerful servents interconnected with a suitable upload and download bandwidth.
\\% What are typical applications?
Applications of peer-to-peer include very popular ones like file
sharing, instant messaging or video calls over the Internet. Other applications
are shared computing, shared workspace environments or information
dissemination, e.g. software updates.
\\% What are possible Applications?
% What are dis-/advantages over the server-client model?
The peer-to-peer design gives several advantages over the client-server model.
In a pure Peer-to-Peer designed network there is no need for a central entity
which makes the network independent of a single point of failure which can also
act as a bottleneck. While in a client-server architecture a growing number of
participants will exceed the resources provided by a central server, in a
Peer-to-Peer network the resources grow with the number of participants. The
lack of a central entity also means that the service can be provided ad-hoc with
the combined resources that each participant contributes to the network without
the need of a dedicated server. Another advantage can be the better
usage of bandwidth. If an information has a single point of entry,
e.g. a video live stream or a software update and a client-server model is used,
the information has to be sent multiple times over the same links, clogging the
bandwidth with repetitive data. A peer-to-peer network could organise this
multicast communication in a way that avoids unnecessary data transmissions. As
mentioned before the peer-to-peer approach eliminates the single point of
failure a central server is, which results in a more reliable service. More
specifically an attack or a failure of a central server disables the service
entirely while a failing servent has minor effects on the whole service.
\\
The next subsection will discuss and classify different approaches in Peer-to-Peer
structures.
\subsection{Types of Peer-to-Peer Architectures}
Several applications using the peer-to-peer paradigm have been proposed and used
within the last three decades. While offering a solution for different services
or resources they also can be classified by the organization of the network
itself. The main difference in architecture being how they organize the
resources or services and handle requests and responses. This subsection will
shortly introduce three main classes of peer-to-peer architectures.
\subsubsection{Unstructured Peer-to-Peer}
\paragraph{Centralized Peer-to-Peer}
The first popular peer-to-peer based application was Napster. It used a
central Server for storing the information where to find which file in the
network. After getting an answer from the central server, a servent could then
directly download the desired file from the origin. This approach can be
generalized as follows. A central server is used for organisation
of the service or resource provided by the network. The requested service or
resource is then provided directly. This architecture is called a
centralized peer-to-peer network.
\\
The main disadvantage is the central server since it contradicts most of the
 advantages of the peer-to-peer architecture, as it acts as a single point of 
failure and is the bottleneck of such a network.
\paragraph{Pure Peer-to-Peer}
In a pure peer-to-peer approach instead of managing the information about the
networks resources or services centrally, every servent manages its resources
locally. Request are send over the network using flooding. A servent sends a
time limited request to all its known neighbors which repeat this action until
the servent able to answer the request is found or the request expires. This
method clearly produces a lot of redundant network traffic which highly depends
on the topology of the network. A peer-to-peer network constructed in such a way
eliminates the need of a central entity is locally organized but has its
bottleneck in the bandwidth which would have to grow exponentially with the
number of participants and requests. The way servents decide which other
participants to select as neighbors is most crucial to the performance of this
type of peer-to-peer networks. The resulting topology's properties have to
match the purpose of the network. The main challenge in constructing such a
network is finding a topology which keeps network traffic at a minimum while not
loosing the decentralized way of network creation.
\paragraph{Hybrid Peer-to-Peer}
There are also applications using a mix of the previously introduced
peer-to-peer architectures. The main reason for this approach is to reduce
the number of messages while still using an unstructured approach. This is
achieved through introducing hierarchies into the network. Superpeers or
ultrapeers are selected dynamically. They are responsible for a number of simple
servants and connected to other superpeers in an unstructured manner. The complexity of such a network
is reduced depending on the number of simple servents the superpeers are
responsible for. There still are no single points of failure although a failure
of a superpeer has a bigger impact on the network than of a servant in a pure
peer-to-peer network. The superpeers can be selected depending on availability,
stability, bandwidth or any other aspect serving the purpose of the network.
\subsubsection{Structured Peer-to-Peer}
While Cyclon is meant for use in unstructured Peer-to-Peer networks this chapter
would be incomplete without mentioning another approach to construct such
networks. Structured Peer-to-Peer networks try to eliminate the cost of
flooding requests while still organising the service decentrally. While in
centralized Peer-to-Peer applications the information about the service is
stored centrally this information is distributed among the servents in the whole
network. This is achieved by distributed hash tables (DHTs). First for every
piece of information a hash value is calculated which determines which of the
servents will be responsible for this information. The servents form a ring
topology. Every servent has one neighbor that is responsible for all
information with smaller hash values and the other for all the information with greater hash
values, than hash values of the information the servent is responsible for
itself. The resulting chain is made to a ring by connecting the servents
responsible for the maximum and minimum hash values. A request is send to an
arbitrary servent on the ring and then forwarded to the servent believed to be
closest to the hash value of the information until the servent with the
information is found. It is clear that this approach greatly depends on the
function calculating the hash values, so that the information is equally distributed
among the servents. In a network with a high churn rate this approach possibly
needs to adjust its hash function to eliminate unequal distribution. This may
produce a lot of overhead compared to unstructured peer-to-peer networks.


\section{Overlay Topologies}
%What is this? 
As pure peer-to-peer networks are managed decentrally,
the nodes themselves have to manage membership. To allow for a peer-to-peer
solution to be scalable, it is desired to have each node be responsible for a
fixed amount of information. This leads to the question how to decide which node
holds which information. Membership management itself requires the nodes to
hold information about other nodes, especially which nodes are currently part of the
network. Since every node knows only a subset of the network, it is crucial to
that each node selects a set of neighbors locally, so that the network profits
from it globally. The nodes and their neighbor lists form an overlay topology
which can, but doesn't have to be dependent on properties of the physical or
the routing topology of the network. The way that nodes are organized in such a
network is also called overlay topology. There have been different models of
what an ideal topology would look like and what properties it would have to
have. This depends on the application of such a network. The properties that are
studied in this thesis are shown in Chapter \ref{chap:properties}. The next
section will discuss the random graph model of Erd\H{o}s and R\'{e}nyi, which
shows to have some of the desired properties.
\subsection{Random Graphs}
% What are they?
Random graph models are network models in which some parameters are fixed but
the networks are random in all other aspects \cite{Bollobas1985}. The
Erd\H{o}s-R\'{e}nyi-Model \cite{erd6s1960evolution} defines a random graph as a
graph G(n,p), where n is the number of nodes and p is the probability that an
edge between to nodes exists. For very large networks in can be shown, that this
model is equivalent to the model G(n,m) where n is again the number of nodes
and m is the number of edges which are placed randomly among the nodes. When
talking about random graphs and their properties, one does not mean single
graphs, but sets of graphs and average properties over those sets. In fact the
properties of random graphs are so fitting for typical peer-to-peer network
applications, that it is desirable for a membership management protocol to
achieve a random graph like topology.
% Definition Erdos renyi model
% properties
% 

\section{Gossiping}
%What is gossiping in general?  
Gossiping in networks is a way of exchanging data in an epidemic way. The basic
idea can be described through the following office image. Workers in an office
meet up pairwise in the break room and exchange information. New information or
gossip is introduced by one worker to another. After the first meeting, two of
the workers possess the new information. At the next meeting in the breaking
room, those two persons talk to other two persons and the information spreads to
two more workers. Of course if the same people pair up again the information is not
spread. But if the pairs are randomly chosen the probability of meeting someone
who already has the new information depends on how far the information has
spread already. On the other hand a person not knowing the new information yet
is far more likely to meet someone with the new information the more it spreads.
This form of information dissemination can also called epidemic. There are
several advantages to this method of information dissemination. First of all new
information is spread fast due to the dissemination's epidemic behaviour. 

\subsection{Classification of Gossiping Algorithms}
Gossiping algorithms can be classified by their information flow. A
push-gossiping algorithm is one, where the initiating communication partner
sends its information to the other without expecting a response. Algorithms
where the initiating partner requests an information from another partner are
called pull-gossiping algorithms. At last algorithms, where the initiating
communication partner sends information to the other and that one responses with
its own information, are called push-pull-gossiping algorithms.


\section{Cyclon} 
%How does Cyclon work? Pseudocode of shuffling algorithms
Cyclon \cite{voulgaris2005cyclon} is a gossiping algorithm for membership
management in unstructured peer-to-peer network. It provides an overlay topology
for decentralized membership management using a push-pull-gossiping approach. Cyclon's goal is to
create an overlay topology with properties similar to random graphs. Each node
holds a changing subset of network participants as neighbors. The
participating nodes periodically exchange neighbor information, making the
overlay dynamic and proactive.  There are two shuffling algorithms for Cyclon, the basic and the
enhanced one.
\subsection{Basic Shuffling}
Basic version of shuffling algorithm is very simple. Each node holds a small
neighbor cache with \emph{c} neighbors. Each cache entry consists of another
peers network address. Every node \emph{P} periodically exchanges its neighbor
information with another node \emph{Q} by performing the following steps:
\begin{enumerate}
\item Select a random subset of \emph{l} $(1 \leq l \leq c)$ neighbors from
\emph{P}'s own cache, and a random neighbor, \emph{Q}, within this subset where
\emph{l} is a system parameter called \emph{shuffle length}.
\item Replace \emph{Q}'s address with \emph{P}'s address.
\item Send the updates subset to \emph{Q}.
\item Receive from \emph{Q} a subset of no more than \emph{l} of \emph{Q}'s
neighbors.
\item Discard entries pointing to \emph{P}, and entries that are already in
\emph{P}'s cache.
\item Update \emph{P}'s cache to include all remaining entries, by
\emph{firstly} using empty cache slots (if any), and \emph{secondly} replacing
entries among the ones originally sent to Q.
\end{enumerate}
The node receiving a shuffle request selects a random subset from its neighbor
cache of a size t ($t \leq l$) and sends it to the initiating node. It then
performs step 5 and 6 with the shuffle request received to update its own cache.
There are two disadvantages to basic shuffling. Since choosing of the node
to exchange information with is done randomly, there is no guarantee that a
specific node is ever chosen for a shuffle. This also implies that a pointer to
a node is never verified by actually communicating with it but just passed along. To deal
with those disadvantages a small change is made to enhance the shuffling.
 
\subsection{Enhanced Shuffling}
Enhanced shuffling adds a new parameter held in the neighbor cache by each node.
To ensure that every node gets shuffled with after a sufficient time has passed
age is added as an parameter to the node pointers. The age is increased every
time a node initiates an exchange between neighbors. Also instead of choosing
the neighbor as the shuffle target randomly, the pointer with the highest
age is chosen. The resulting shuffle operation takes the following steps:

\begin{enumerate}
\item Increase by one the age of all neighbors.
\item Select neighbor Q with the highest age among all neighbors, and l-1 other
random neighbors.
\item Replace Q's entry with a new entry of age 0 and with P's address.
\item Send the updated subset to peer Q.
\item Receive from Q a subset of no more than i of its own entries.
\item Discard entries pointing at P and entries already contained in P's
neighbor cache.
\item Update P's cache to include all remaining entries,by firstly using emty
cache slots, and secondly replacing entries among the ones sent to Q.
\end{enumerate}

This enhancement, additionally to solving the problem that a node is never
chosen as a shuffling partner, also limits the time a pointer lives. A pointer
can only seize to exist, if it is chosen as a shuffle target, is exchanged
between to nodes that both know it or when it is lost due to a communication
failure. The implementation in this thesis uses the enhanced shuffling method
especially because of this feature.

