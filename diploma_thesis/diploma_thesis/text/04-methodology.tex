\chapter{Measurement Methodology and Implementation} 
\label{chap:methodology}
This chapter describes the architecture of the experiments and the environment
in which it is build. It starts with the physical network on which the
experiments are conducted, the DES-Testbed. The next section is about the Layer
3 routing algorithm used for wireless experiments, which is the open link state
routing deamon. The implementation of Cyclon is then described and
explained in the next section and all other environment specifics are given in
the last section of this chapter.

\section{DES-Testbed}
%What is this? How does it work? Why do I use this?
The DES-Testbed is a wireless adhoc mesh network environment developed and
deployed at Freie Universität Berlin to test network protocols for embedded
systems under real conditions. It consists of currently 51 mesh routers each
with 3 wireless network interfaces and 2 ethernet interfaces. The mesh routers
are placed in mostly seperate rooms at the Institute of Computer Science, the
Institute of Mathematics and the Intitute of Physics at the Freie Universtät
Berlin. They are connected through an ethernet network to the main entry point,
which in the experiments described in this thesis is also used as a statistics
server for gathering experiment related information. Further the wireless mesh
network is used to introduce network faults like packet loss and a
dynamic layer 3 routing topology to the experiments. The network is also used
for experiments in ethernet environment to obtain comparable results without the
afore mentioned network faults.
%

\section{OLSR/B.A.T.M.A.N.} %What? How? Why?
%what version of olsrd do i use?
The underlying routing protocol used for the experiments is the OSLR deamon
\cite{Tonnesen2004}.
It is the implementation of the optimized link state routing protocol (OLSR) as
described in \cite{RFC3626}. It creates a decentralized multi-hop routing topology for
wireless mesh networks. It is used successfully in large metropolitan area mesh
networks like the Athens Wireless Metropolitan Network or the Berlin based Freifunk.net.
The basic idea behind it, is as follows. To discover its immediate one-hop and
two hop neighbors, a node sends HELLO-messages paired with one-hop neighbor
information. It then chooses a set of multipoint relays (MPR) within its one-hop
neighbors with the best link quality to all of its two-hop neighbors. These are
used for optimized flooding of control packages. Every node also has an MPR
selector list, which holds information about all nodes that chose that node
as an MPR. The nodes that are selected as an MPR will now send and forward MPR
selector lists using TC-messages. The OLSR protocol is proactive, which in this
case means, it changes its routing topology regularly, but not immediately
whenever a new node appears or an old link to a node is unstable.

\subsection{Txtinfo and Jsoninfo Plugins}
The OLSR daemon created by \cite{Tonnesen2004} also offers the txtInfo plugin
to access neighbor, topology and other information from a running OLSR
daemon. It is used by the JsonInfo plugin, which again creates a Java
interface for access to that information. This is used to obtain information
about the physical topology of the network and for bootstrapping purposes as
described later.

\section{Implementation of Cyclon in JAVA} %UML-Diagramm
For the purpose of this thesis, Cyclon has been implemented in JAVA to perform
experiments on the DES-Testbed. The program consists of the statistics server
used to collect all experiment data and the Cyclon client itself. The later
program simulates a given number of Cyclon clients using different ports on one
network interface. It can communicate with Cyclon clients on other testbed
nodes as well as with clients running on the same node. Each client has its own
neighbor cache and creates a thread that initiates shuffle periodically. To
simultaneously receive shuffle requests and responses, every clients also has a
thread dealing with listening to its assigned port and processing incoming
requests and responses. The neighbor cache of each Cyclon client can hold a
limited amount of neighbor pointers, each consisting of IP address, port and
age. Every time a client receives a shuffle request or an answer, it sends its
updated neighbor cache to the statistics server by invoking a web method.
\subsection{The Bootstraping Process}
Bootstrapping considers the way a new client joins a network for the first time.
The Cyclon client has to have at least one client to start exchanging neighbor
information and get connected to the network. There are several ways to get the
first node to communicate with. One can use a previously known network
participant from an old cache, try to contact randomly generated addresses,
until a node responds or use a well known bootstrap server. Bootstrapping is
completely independent of the shuffling protocol and poses the same problem to
all other peer-to-peer networks.

In this thesis the bootstrapping is done in a way, that works only in this
specific environment. Whenever the neighbor cache of a client is empty, which
mainly occurs in new clients, the client uses the jsoninfo plugin to obtain a
random physical neighbor over the wireless network device. It then requests the
last known active port of this node from the statistics server, to complete the
neighbor information. The age is set as new and the client can start shuffling
with the obtained node. Experiments that run on the Ethernet network of the
DES-Testbed rather than on the Wifi network managed by the OLSR daemon also use
OLSR for bootstrapping. This is done so that the starting conditions for both
experiment modes are similar for sake of comparison. In order to do this, the
IP address of the wireless network interface delivered by the bootstrapping
process is altered to address the same physical node through its Ethernet
interface. As there is no guarantee that this bootstrapping process creates a
connected starting topology, only experiments where this is the case are
considered. In addition multiple clients on the same physical node are
interconnected at start up. The resulting starting topology is a random graph
with a fixed number of nodes, edges and the premise that the graph is composed
of a single weakly connected component.

\subsection{The Cyclon Client}
The Cyclon peer implementation (see figure \ref{fig:CyclonPeerUML}) consists
of the neighbor cache, a thread to listen for and process incoming messages and a thread to initiate shuffling
periodically. The neighbor cache is a list of neighbors. On initialisation a
peer is given a network interface, a port and the IP address and port of the
statistic server. Internal parameters are cache size, shuffle length, shuffle
period and socket timeout. Internal parameters that are constant during all
experiments are:
\begin{itemize}
  \item \emph{socket time out} is set to \emph{3000 milliseconds}, which is
  several magnitudes larger than the largest return time in the network and also
  equal to the shuffling rate
  \item \emph{shuffling rate} is set to \emph{3000 milliseconds}, this gives the
  statistics server an average of 5 milliseconds to process each change in the
  network
\end{itemize}
It creates a UDP network socket for sending and receiving Cyclon packets over
the network. It can also be given a seed or bootstrapping node. If not given a
bootstrapping node or whenever the neighbor cache is empty the node initiates
the bootstrapping process to obtain at least one valid neighbor. Each peer then
waits for a random time within 3000 milliseconds, so that the peers shuffle
asynchronously from the start. This is done to ensure that the statistics
server has an even load balance, since it is the bottleneck of the experiment
set-up. This closes the initialisation phase.

The peer then starts a thread for periodically sending shuffle requests and a
thread to listen to any incoming messages. The shuffle process starts with
incrementing the age of all neighbors in the neighbor cache. It then chooses
the neighbor with the highest age as the current target. The peer picks and
tags a random subset of the neighbors in its neighbor cache. It generates a
random shuffle ID and creates a Cyclon shuffle request packet. The packet, as
depicted on figure \ref{fig:CyclonPacket}, includes the generated shuffle id
and a number of neighbors limited by the shuffle length.
The first neighbor entry in a shuffle request package is always the shuffle
request sender. A neighbor entry consists of the IP address, the port and the
age of the pointer. The peer sends the packet to the current target node and
waits for a response. If the listening thread receives a response it notifies
the shuffle thread, which then waits for the shuffle period before restarting
the shuffle process. If, after the socket timeout has ran out, there is still
no response, that means that either the request or the response packet got lost
or exceedingly delayed. The peer deletes the current target from its neighbor cache
and restarts the shuffling process.

The listening thread waits for incoming messages using the shuffle ID to tell
between requests and responses. If a response is received the shuffle thread is
notified and the response is processed as follows. First any already known
peers are deleted from the response. Then any empty slots in the neighbor cache
are filled with the new neighbors. If there are still neighbors left in the
response, neighbors that are tagged, are replaced with those from the response.
Remaining tagged nodes are then untagged. The updated neighbor list is then send
to the statistics server.

A request is processed as follows. First a response is generated by picking and
tagging a random subset of neighbors from the neighbor cache. Those neighbors
are send back with a shuffle response packet, containing the shuffle ID from
the request and the randomly picked neighbors along with their age information.
Then the request is processed in the same way a response is treated. 
%DES TESTBED nodes as bottleneck
\begin{figure}[ht]
	\centering
	\begin{bytefield}{32}
		\bitheader{0-31}\\
		\wordgroupr{Header}
			\bitbox{32}{message ID}
		\endwordgroupr \\
		\wordgroupr{ First neighbor entry}
			\bitbox{32}{neighbor IP address}\\
			\bitbox{16}{neighbor port}\\
			\bitbox{32}{age}
		\endwordgroupr \\
		
		\wordbox[lrt]{1}{more neighbors} \\
		\skippedwords \\
		\wordbox[lrb]{1}{} \\
	\end{bytefield}
	\caption[Packet Format]{Packet Format of Cyclon in this Implementation}
	\label{fig:CyclonPacket}
\end{figure}

\begin{figure}
	\centering
	\includegraphics[width=\textwidth,height=\textheight,keepaspectratio]{./graphics/CyclonClientUML.pdf}
	\caption{Simplified UML-Class Diagram for the Cyclon Peer}
	\label{fig:CyclonPeerUML}
\end{figure} 
In all experiments the shuffling is done asynchronous, but at a fixed rate,
which is magnitudes bigger than the common latencies of the underlying network.
This prevents Cyclon packets to be received out of order and other unwanted
effects. The rate is set to 3000 milliseconds for all experiments and
experiments run for at least 60 min allowing each node to initiate a shuffling
algorithm for 1200 times. The rate in which the network evolves is also
carefully picked, to ensure that the statistics server can handle incoming
information.
\FloatBarrier
\subsubsection{The simulation modes}
The Cyclon client can be run in two experiment modes. The static mode starts a
given number of peers, assigns the same IP address and a different port to each
and lets them run until the program is terminated. The churn mode starts a
given number of container threads, which each starts one Cyclon peer giving them
a unique port. The container threads calculate a random lifetime for each peer
and kill them after their lifetime has depleted. After that, they repeat
starting new peers until the program is terminated or the given port threshold
is exceeded.
\subsubsection{The output mode}
The output mode is used to obtain data surveyed by the statistics server, it
invokes the output web method of the server and writes the results into a Gephi
file.  
%Cyclon packet diagram 

\subsection{The Statistics Server}

The statistics server is used to obtain experiment data and create an outputfile
in the Gephi file format. It provides a web service to obtain the data from the
clients and to create an Gephi readable XML file.  Each node sends routing
topology and the Cyclon neighbor list to the statistics server
through web methods called sendTopology and sendList. The information then is
processed by the statistics server by storing the node along with its neighbor
information in a node object. It processes the information by storing the
sender as a node updating its neighbors as edges After the experiment is
finished the server serializes the stored objects to an XML stream and writes
it to a file in the Gephi file format described below.

The statistics server application stores all relevant information for an
experiment during its execution. On initialization it creates a collection of
nodes to store the required information about each node as they are born,
change neighbors and eventually die. The server then publishes its web service,
which holds methods to access the server's objects. The main web method is used
by the nodes to update information whenever their status changes. This might be
their birth or the change of their neighbor cache. The node object stores the
information of its first appearance, the last time it got updated and about all
of its edges and at which times they existed. If the node didn't get updated for
a specific time it is marked as dead. Should the node ever be updated again it is
reset to alive as if it wasn't dead. All timestamps are stored as nanoseconds in
relation to an arbitrary point in time. The timestamps are given as the
statistics server gets the information, so that delays on the network could
distort the results.
The statistics server is implemented as a JAVA web service providing an
interface for storing all information concerning the graph topology during an
experiment. The web service also provides a bootstrapping service. The
bootstrapping service returns an active port for a given IP address. This is
useful in experiments where a Cyclon client starts with an empty neighbor
cache, obtains an IP address from the underlying routing protocol, but does not
know which port to use. The interface specification is shown below
as an UML class diagram and the web methods included are described below.
The way the statistics server is implemented allows to reuse it for information
gathering in other experiments. 
%UML class diagram of statistics server class

\section{Experiment Evaluation Environment}

\subsection{Gephi and GEXF File Format}
To calculate most of the topology parameters, the network analyzing tool Gephi
is used. The tool provides a graphical interface to visualize and analyze Graphs
given in the gexf-format. The GEXF File Format is a simple XML-style file
format. An example is given in the listing \ref{lst:gephiFormat}\\

\begin{lstlisting}[caption=Example of a graph in GEXF file
format,label=lst:gephiFormat, language=XML] 
<?xml version="1.0" encoding="UTF-8"?>
<gexf xmlns="http://www.gexf.net/1.2draft" version="1.2">
    <meta lastmodifieddate="2009-03-20">
        <creator>Gexf.net</creator>
        <description>A hello world! file</description>
    </meta>
    <graph mode="static" defaultedgetype="directed">
        <nodes>
            <node id="0" label="Hello" start="0" end="300"/>
            <node id="1" label="Word" start="0" end="300"/>
        </nodes>
        <edges>
            <edge id="0" source="0" target="1">
            	<spell start="0" end="120"\>
            	<spell start="200" end="250"\>
            <\edge>
        </edges>
    </graph>
</gexf>
\end{lstlisting}
